\documentclass{article}

\usepackage[backend=biber, style=authoryear-icomp]{biblatex}
\addbibresource{CS4182 Group 5 Project References.bib}
\usepackage{titlesec}
\usepackage{titling}
\usepackage[margin=1.25in]{geometry}
\usepackage{hyperref}

\titleformat{\section}
{\huge}
{}
{0.25em}
{\filcenter\bfseries}

\titleformat{\subsection}
{\bfseries}
{\hspace{-.25in}}
{0em}
{}

\titleformat{\subsubsection}[runin]
{\bfseries}
{}
{0em}
{}[---]

\titlespacing{\subsubsection}
{0em}{0em}{0em}

%\renewcommand{\maketitle}{}

\author{Edison Cai, Sergiu Mereacre, Bayan Nezamabad, Jack O'Brien}
\title{The State of Autonomous Driving: Progression and Future Adoptability}

\begin{document}

\maketitle

\section{Current Safety and Efficiency of Autonomous Driving}

Autonomous driving and autonomous vehicles are currently some of the most heavily studied and publicly followed technologies in the automotive realm. This technology could greatly improve safety, efficiency, and mobility by replacing the driver and relying on vehicles to navigate themselves through traffic. In order to accomplish the vision of autonomous driving, many technical challenges need still to be solved. But there are also many non-technical challenges to be considered on the path toward a vision of autonomous vehicles. The legal challenges being among the most critical. As automobiles are becoming more and more self-reliant in making mission-critical driving decisions, public policies, and technical standards need to be revisited to prepare courts and the public for the new realities of traffic with autonomous vehicles.[\textcite{beiker2012legal}]

\subsection{Automatic and Autonomous Driving Assistance Systems}

Firstly, it is important to note there is a difference between fully automatic and autonomous driving systems. The question “Does the driver have to stay in the loop, if he is driven by a car, which is completely guided by a computer?” has to be answered. In any case, from an outside view the behaviour of the car should take application dependent, definite rules into account such as road traffic regulations. In other words the vehicle has to act autonomous in terms according to moral law. In the civil area the effective moral law are road traffic regulations, which aim for collision avoidance and physical inviolability. [\textcite{horwick2010strategy}]
\bigbreak
The study type of active system used is known as the driver assistance system (DAS). This system helps the driver and provides a safer driving experience with technology that provides automated or adaptive driving functionalities. According to the Society of Automotive Engineers (SAE), there are six levels of vehicle autonomy[\textcite{curiel2019towards}]:
\bigbreak
\begin{itemize}
\item Level 0: no automation. The driver has full control and performs all driving functions at all times.

\item Level 1: driver assistance. The driver has full control, but the vehicle provides assistance for one or more driving functions, e.g., electronic stability control or assisted braking.

\item Level 2: partial automation. The driver has primary control over the vehicle, but the vehicle can take full control of more than one driving mode, e.g., steering and acceleration/braking in combination.

\item Level 3: conditional automation. The automated driving system has primary control and performs all driving functions under certain conditions. Human driver intervention is requested in many driving modes.

\item Level 4: high automation. The automated driving system has almost full control over the vehicle. Human driver intervention is needed only in some driving modes.

\item Level 5: full automation. The automated driving system has full control over the vehicle. No human driver intervention is needed.
\end{itemize}

\subsection{Safety Features(CAV’S)}

With advances in sensing, machine learning, and computing systems, autonomous driving applications have become feasible and ready to use. This, combined with the use of communication technologies like dedicated short range communications (DSRC), autonomous vehicles can also communicate with each other, pedestrians and the infrastructure in the environment around them. We refer to vehicles with such capabilities as connected and autonomous vehicles (CAVs). As the number of applications and levels of autonomy increase, the complexity of CAVs continues to increase.
\bigbreak
 CAV systems consists of large numbers of sensors that sense its environment. The data from these sensors are fed into a computing system that runs a variety of software components. These software components process the data, perform calculations and control actuators to achieve autonomous driving applications. The design and modelling of various software components that constitute an important step in the development of CAV systems. This step also includes defining the interactions of these software components with each other and the hardware components. The next step in the development process involves the deployment of these components onto a real system.[\textcite{bhat2018tools}]

\subsection{Problems to Overcome}

 Most of the available computer vision applications are based on cameras that can only be used under the condition of normal light and clear weather, which makes most of the state-of-the-art models not suitable for night images. Traffic safety statistics show that 51.1\% of the U.S. fatal crashes happened at night (from 6 pm to 6 am), especially in rural areas with low illumination. Enhancing  night images for a clear traffic environment is a step in the right direction for traffic safety and should be incorporated into ADASs and CAVs.
 \bigbreak
However, the outline and appearance details of traffic participants are easily blurred at night, making it hard to tell apart target objects from the background. Therefore, restoring the details of low-light images is a hard task, especially for the low-light images. One method to solve this problem is histogram equalization (HE). HE makes brightness better distributed on the histogram, which can be used to enhance the local contrast without affecting the overall contrast. Another method is gamma correction, which increases the brightness of dark regions by compressing bright pixels. However, these approaches only focused on the night situations with external light sources (e.g., urban roads with street light). There is a need for an image enhancement model that can be used in darker areas because crashes and accidents are more likely to happen while driving in rural areas at night without streetlights[\textcite{li2021deep}].

\subsection{Crashes and the Responsibility of the Manufacturer VS the Owner}

Should we hold the manufacturers responsible for any crash caused by the vehicle? If there is some design decision in the system, which causes an accident in certain situations, or perhaps a flaw, they probably knew about or should have discovered. Why should they not have to take responsibility?
\bigbreak
On the other hand, there is question of whether we should try to promote the development of autonomous cars to begin with. In other words: should we try to design the liability for autonomous vehicles in such a way that it promotes their continuous development and improvement? Should such vehicles be allowed on our streets? If there are good moral reasons for finding the development and introduction of autonomous cars to be desirable, this can produce a moral obligation for the state to fashion the legal responsibility for crashes of autonomous cars in a way which helps the development and improvement of autonomous cars.
\bigbreak
There are many arguments which can be made in favour of or against the introduction of autonomous cars. Possible problems include privacy issues (Glancy 2012) and environmental harm from fully-autonomous vehicles, as these could lead to more vehicle-miles travelled (Elkind 2012). On the positive side, the introduction of autonomous cars might among other things enable the physically impaired, disabled or elderly to drive their own vehicles (Howard 2013).
\bigbreak
The development and widespread use of autonomous cars could cause a reduction of accidents, it could therefore save lives. Even if we are talking of a relatively small improvement like a reduction of 5\% it would save hundreds of lives a year in countries like the US in which deaths in road accidents go into the tens-of-thousands. 
\bigbreak
An alternative would be to hold the users of autonomous cars responsible for possible accidents. One way of doing this would be for the driver to pay attention to traffic and take control of the car when necessary. The liability of the driver in the case of an accident would be based on his failure to pay attention and intervene. The problem is that Autonomous vehicles would lose much of their use. It would not be possible to send the vehicle off to look for a parking place by itself or call for it when needed. 
\bigbreak
As long as there is some evidence that a system in which people must intervene would do noticeably better in terms of number of accidents than one in which autonomous vehicles are left to themselves there is much to be said in favour of such a duty. If the introduction of autonomous vehicles reduces accidents by fifteen percent, and a duty to intervene for the “driver” would lower the death rate by another fifteen, that would seem to create a moral obligation on drivers to be on the lookout for possible failure. Also, it would also give the technology an opportunity to develop gradually. Autonomous driving could slowly evolve, going from the current level of automation through a number of intermediate stages to fully autonomous cars. On the downside, self-driving cars would, in such a scenario, not be useable by physically impaired, disabled or elderly people[\textcite{hevelke2015responsibility}].


\section{Progression of Autonomous Vehicles and Supporting Technologies}

Since the constituents that make up autonomous vehicles are vast, we will be focusing on some key elements and how they are progressing. Computing power paired with machine learning is exponentially improving the capabilities and safety of every vehicle and plays a vital role in receiving permits and permission from regulators to operate. Declining battery costs will drive down production costs which directly translates to savings for potential customers and drive-up the rate of adoption. The gap between the total cost of ownership of battery powered electric vehicles and regular cars can be overcome if the cost of batteries drops to around
€150/kWh. [\textcite{van2011energy}] The synergy between batteries and software will be integral in keeping stable ranges and temperatures [\textcite{ali2019towards}]. Progress is also being made in establishing charging infrastructure that supports driverless vehicles using sustainable sources of energy such as wind and solar [\textcite{nunes2015day}]. These aspects are vital to the success and widespread adoption of these vehicles.  

\subsection{The Importance of AI and Making Decisions}

We are now in an age where the progress in technology is accelerating. Along with the help of AI, the world will be a vastly different place in a few years time "The impact of AI technologies can be even more profound than that of both the Industrial and digital revolutions put together, as it holds the potential to affect practically all tasks currently performed by humans,"[\textcite{makridakis2017forthcoming}]. Autonomous vehicles will come as the result of many rapidly developing technologies converging into a single product. These include neural networks, machine learning, deep learning, battery engineering, renewable energy, software engineering, and much more. 
\bigbreak
To get to this point, every autonomous vehicle must exhibit some intelligence and ability to make its own decisions if it wants to navigate the constant changing environment. It needs to accept several inputs from the outside such as geometric shapes and signs and output an action that will result in safe and smooth driving. It needs to distinguish between what needs to be avoided, predict the paths of different objects, anticipate where everything will end up due to physics and be able to apply reasoning behind every decision [\textcite{pomerol1997artificial}]. There is no realistic method for developers to manually code every situation as the effort that is demanded is gargantuan.  
\bigbreak
Artificial Intelligence works in this context because as you feed it more data, it will gradually learn patterns which is not too different from how the human mind learns. Researchers have attempted to imitate the human brain's learning process by creating networks of artificial neurons. To have it recognise faces, they show the network training examples, compare the actual activity of its outputs with the desired activity before tweaking the weight of each connection to reduce the error and subsequently training the network [\textcite{hinton1992neural}]. In a similar fashion, we also learn through experience and error correcting. This is the ideal method for training autonomous vehicles and handling edge cases that may otherwise not be caught. To best train autonomous vehicles, you must have high quality, real world data which must be retrieved in an efficient manner and have substantial quantity. Errors must be labelled to nudge the neural network towards higher accuracy.

\subsection{Image-Recognition Training}

Many companies have their own approaches at attempting to solve autonomous driving. For example, Google's Waymo team builds "three-dimensional maps that highlight information" using Lidar technology, sending out laser pulses to measure ranges. They have already released their robotaxi service in the Phoenix Metropolitan Area, Arizona, United States of America \textcite{waymo}. For our discussion, we will take a closer look at Tesla's Full Self-Driving team and how they leverage data clips gathered from their vehicles' suite of cameras. At the moment, their team manually label video clips, identifying objects such as road markings and signs to train the neural network. At first, they only labelled individual frames from each of the cameras which can be time consuming. However, by using visual Simultaneous Localization and Mapping or visual SLAM, the cameras can also build three-dimensional maps, allowing staff to label a single three-dimensional scene as opposed to thousands of two-dimensional frames, drastically improving efficiency [\textcite{durrant2006simultaneous}].
\bigbreak
Unfortunately, scaling this approach will be financially demanding as humans are expensive compared to computers. Soon, a big central processing unit will make it so that humans are needed less, and the AI will automatically label points based on interventions from clients. Andrej Karpathy humorously named this objective "Mission Vacation", where the final goal is to have every human employee go on a vacation while the neural network continues its training. An example of progress made in recognition technology can be found in the paper [\textcite{karpathy2014large}], where researchers are able to use neural networks to identify sports in video clips. By training the network using "1 million YouTube videos belonging to a taxonomy of 487 classes of sports", even with weak or inaccurate annotations in the videos' metadata, it was still able to predict what sport was playing with high levels of confidence.
\bigbreak
Another fundamental constraint to the improvement of autonomous driving technology apart from the data set is computing power. To process all of these images and retrain the network constantly, huge computing power is required. Through the use of supercomputers, training can be done more efficiently.
Right now, Tesla is using a huge central computer called Dojo to train the neural network with no limitations. It can process a lot of data with impressive speed. Thus begins the march of nines, where safety can never be 100\% assured but we can converge to it as closely as possible. Once Dojo is confident in handling most situations and edge cases, software engineers can then refactor it, making it more efficient to make it compatible with the computers and memory held in autonomous vehicles. It has now also gotten to a point where the neural network KNOWS what its weak points are and can specifically request the driving network to retrieve specific edge cases so that it won’t be getting redundant information that it’s already adept at handling. 

\subsection{Hardware Requirements}

Computer hardware needs to be able to handle the processing of videos and logic and prediction algorithms. This means that manufacturers must calculate the amount of processing power that is needed and then implement a computer of suitable specifications in the vehicles to avoid the pain of upgrading them should be prove incapable of handling the computations and the storage of data. It also must be efficient enough so that it won’t absorb too much drivable range from the vehicle. 

\subsection{Batteries}

Electric cars are getting more efficient and battery costs are declining (following a predictable cost curve) , with the infrastructure also improving rapidly. Solar and wind is spreading as their value proposition becomes more compelling. Jeff Dahn’s continued research in various battery chemistries have yielded promising results. Using a certain mix of nickel, cobalt, aluminum, etc., a battery that lasts a million miles or more with minimal degradation is possible.  

\subsection{Decentralised Power, Renewable Energy}

Decentralised grids will make sure that charging stations aren’t reliant on a single source of power. To charge the cars, they might have to do it autonomously as well. Experiments are being done with charging robots that can detect ports and plug it in. Battery swapping? Improved wireless charging technologies? Software should also change how a battery operates, working in its heating and cooling components intelligently. 

\subsection{Software in a World of Autonomy}

Software will also become a big part of autonomous vehicles as customers will now have free time on their hands to enjoy the screens in the middle. It might need to connect to the Internet via 5G or satellite to stream content at acceptable speeds. Receivers are becoming smaller and more efficient. The software experience should be smooth and will require suitable computer hardware to run it. Its design should be easy to navigate. It should take inspiration from strides made in the mobile phone and video game industry. Avoid obsolescence by implementing capability to receive over-the-air updates. The actual autonomous driving capabilities should also be improved using over-the-air updates instead of manually updating in an updating warehouse to save land space and resources and labour. 

\subsection{Building Trust Using Data}

What about the progress in building the trust of the general public? Companies must strive to prove that their vehicles will have the chance of failure. Governments will have to put safety regulations in place. What regulations have been placed on other mode of transport in the past? Companies can build confidence by making public how their technology works and how it’s improving. Track record will be highly important. The data it collects to train the autonomous network must be protected and be anonymous and there are encryption methods. "Trust is an
evolving and fragile phenomenon and can be destroyed much more quickly and easily than it can be created." \cite{hengstler2016applied}

\printbibliography
\end{document}